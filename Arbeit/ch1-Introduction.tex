\chapter{Einleitung}

Wasserverteilungssysteme (engl. Water Distribution Networks, WDNs) gehören zu den wichtigsten Infrastrukturen
 einer Gesellschaft, in welcher der Anspruch besteht, Trinkwasser in ausreichender Menge und Qualität auf die
 Haushalte zu verteilen. Da der Hauptteil dieser Netzwerke jedoch unter der Erde vergraben liegt, sind WDNs
 anfällig für plötzliche sowie schleichende Schäden, die häufig ohne direkt sichtbare Indikatoren
 auftreten können. Das Wasser kann sich hierbei verunreinigen und geht in Massen verloren. Weltweit beträgt
 der jährliche Wasserverlust, welcher vor allem durch solche Lecks vorangetrieben wird, 126 Milliarden
 Kubikmeter. Auch wenn es in Deutschland nur 5\% des gesamten Wasseraufkommens sind, kann es in manchen
 Extremfällen bis über 50\% gehen. Der finanzielle Schaden allein beträgt weltweit knapp 40 Milliarden USD
 jährlich [1]. Doch nicht nur aus finanzieller Sicht ist dies dramatisch. In Zeiten des Klimawandels muss auf
 eine effiziente Nutzung endlicher Ressourcen Acht gegeben werden.

Das Erkennen von Lecks muss demnach schnell und akkurat sein. Alleine das Freilegen der Rohrleitungen kostet viel
 Zeit [Quelle], weswegen die Rate fälschlicher Erkennung niedrig gehalten werden sollte. Auch wenn manche Lecks
 einfach zu erkennen sind, beispielsweise durch sichtbar austretendes Wasser oder spürbaren Druckverlust, ist in
 den meisten Fällen der Wasserverlust unterirdisch. Um solche Arten zu erkennen, können Sensoren installiert
 werden, die den aktuellen Druck in den Rohren messen. Da das Installieren von Drucksensoren unter der Erde
 ebenso einen größeren Aufwand bedeuten, haben [Q1] und [Q2] eine Methode entwickelt, eine optimale
 Sensoren-Verteilung zu ermöglichen.

\section*{Inhalt der Arbeit}

Thema dieser Bachelorarbeit ist es, die Sensordaten zu analysieren und durch ein datengetriebenes Modell
 potentielle Lecks in WDNs frühzeitig zu erkennen. Hierfür ist die Arbeit in drei Kapitel aufgeteilt:
 Zuerst wird auf die theoretischen Grundlagen der Anomaliedetektion in WDNs sowie ausgewählter Modelle und
 Ideen des Maschinellen Lernens eingegangen.
In diesem Zuge wird auch untersucht, welche Metriken zur Evaluation von Detektions-Modellen geeignet sind.
Wie zuvor erwähnt, ist dies kein triviales Thema, da neben der Erkennungsrate auch auf eine niedrige
 Fehlalarm-Quote sowie eine kurze Zeit vom Auftreten der Lecks bis zu ihrem Erkennen geachtet werden muss.

Weiterführend wird auf die Methodik eingegangen. Hierfür wird zunächst der Datensatz analysiert gefolgt von
 der Implementation des Erkennungsmodells. Grundlage für ein solches Modell bilden sogenannte digitale Zwillinge,
 welche digitale Repräsentationen realer Abläufe sind; Die Sensorwerte werden simuliert und mit den echten
 Werten verglichen. Eine zu hohe Diskrepanz kann dann als Problemfall gemeldet werden. In der Vergangenheit
 wurden diese simulierte Netzwerke häufig als hydraulische Systeme gelöst, also einer Menge an spezifisch auf
 das WDN angepassten, mathematisch-physikalischen Gleichungen. Diese fordern jedoch in der Modellierung hohe
 Expertise und besitzen während der Kalibrierung eine starke Komplexität durch viele Freiheitsgrade. Somit
 werden häufiger Methoden aus dem Bereich des Maschinellen Lernens verwendet, welche die Gesetze der Hydraulik
 nicht kennen müssen und anhand großer, problem-spezifischer Datenmengen selbst lernen.

Zuletzt werden die Ergebnisse der Methoden vorgestellt und anschließend diskutiert. Hierbei wird auf die
 Probleme lokaler Optima sowie die Realisierbarkeit der Modelle eingegangen.
