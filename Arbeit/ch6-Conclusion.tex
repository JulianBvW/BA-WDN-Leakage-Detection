\chapter{Zusammenfassung}

In dieser Arbeit wurde in die Problemstellung der Leck-Detektion in Wasserverteilungsnetzwerken eingeleitet under
 wichtige Kriterien hervorgehoben sowie verschiedene Ansätze des maschinellen Lernens für die Eignung als
 Detektions-Algorithmen getestet. So ist für diese Technologien wichtig, schnelle und präzise Vorhersagen
 treffen zu können. Dafür wurden die Ansätze in verschiedenen Szenarien mit variierenden Bedingungen trainiert
 und evaluiert. Die Ergebnisse zeigen, dass ein Ansatz, indem ein Ensemble von Regressionsalgorithmen ein
 virtuelles Netzwerk prognostiziert auf realistischen Daten nicht nur für die Optimierung der Metriken, sondern
 auch für die echte Anwendung am besten funktioniert.

Zudem wurden für den Erfolg wesentliche Hyperparameter in der Feinabstimmung identifiziert und auf ihren Einfluss
 analysiert: So ist es wichtig, den Threshold dynamisch nach der Tageszeit auszurichten und die Anzahl an Knoten,
 die Alarm schlagen müssen, damit das gesamte Modell einen Zeitpunkt als Leck prognostiziert, niedrig zu halten.
 Auch das richtige Auswählen einer Aktivierungsfunktion für MLPs kann das Ergebnis stark verbessern.

Als letztes wurde gezeigt, dass das Erweitern der einzelnen Datenpunkte um Informationen von vergangenen
Zeitpunkten helfen kann, mehr Sicherheit in der Vorhersagen zu schaffen.

Abschließend lässt sich sagen, dass vor allem das Regression-Ensemble einen vielversprechenden Ansatz für die
Leck-Detektion bietet, welche auf größeren Netzwerken weiter getestet werden sollte.
