\chapter{Anomaliedetektion}

Für die Detektion von Lecks in WDNs gibt es mehrere Ansätze, welche sich in aktive und passive
 Strategien einteilen lassen \cite{lee2021development}.

\begin{itemize}
    
    \item Die \textbf{aktiven Verfahren}, auch hardwarebasierte Verfahren, sind Strategien, die mittels
     spezialisierter Technik aktiv nach Brüchen in den Rohren suchen. Hierfür könnten zum Beispiel
     Schallgeneratoren oder Kameras benutzt werden. Dabei steht jeder Suchvorgang in der Regel für sich
     und bildet keinen Verlauf ab. Zudem sind aktive Verfahren sowohl Zeit- als auch Ressourcenintensiv.

    \item Dahingegen sind \textbf{passive Verfahren}, oder auch modellbasierte Verfahren, darauf ausgelegt
     ein kontinuierliches Bild über den aktuellen Zustand des Netzwerkes zu geben und bei Anomalien Alarm zu
     schlagen. Hierfür wird die bereits erwähnte Technik der digitalen Zwillinge angewendet, indem das reale
     Netz, welches durch eine Menge an Sensoren durchgehend überwacht wird, durch ein virtuell simuliertes
     Netzwerk erweitert wird. Mit der Annahme, dass die Simulation den Normalzustand, also eine leckfreie
     Version des Netzwerkes, abbildet, können aus zu hohen Differenzen Anomalien abgeleitet werden. Die
     Simulation kann auf zwei Wege erstellt werden. Die Hydraulic Model Based Verfahren nutzen spezifisch
     kalibrierte, hydraulische Gleichungen. Sie berücksichtigen alles von der genauen Struktur und Höhenlage
     über Material bis hin zur Rohrdicke und Alter. Die resultierende Menge an Gleichungen ist daher sehr
     komplex und sensibel gegenüber Ungenauigkeit in der Beschreibung. Auf der anderen Seite stehen die
     Hydraulic Measurement Based (oder auch datengetriebenen) Verfahren. Diese nutzen maschinelles Lernen um
     mittels Langzeitdaten des Netzwerkes Vorhersagen zu treffen. Dadurch ist kein Vorwissen über die
     hydraulischen Eigenschaften des Netzwerkes nötig, da das notwendige Wissen von den Algorithmen gelernt wird.
    
\end{itemize}

In dieser Arbeit geht es um die datengetriebenen, also die passiven, auf vorangegangenen Messwerten
 basierten, Verfahren. Ein solches Modell funktioniert in zwei Schritten. Im ersten Schritt wird das
 virtuelle Netzwerk simuliert. Hierfür wird für jeden echten Sensor ein digitaler Sensor erstellt. Dieser
 schätzt den eigenen Druckwert anhand der aktuellen Druckwerte aller anderen Sensoren. Möglicherweise können
 auch direkt vorangegangene Messwerte in diese Vorhersage mit eingebaut werden. Wird dies für jeden Sensor
 gemacht, entsteht ein virtuelles Netzwerk, bei dem jeder digitale Sensor kein Wissen über seinen realen Wert
 hat. Im nächsten Schritt wird dann zuerst die Differenz der Netzwerke berechnet. Ist momentan kein Leck im
 WDN, so sollten diese Differenzen klein sein, während Lecks zu höheren Differenzen führen. Hier gilt es nun,
 einen Threshold zu finden, ab dem eine Differenz zu hoch ist, um im Normalbereich zu liegen. Ist dieser
 abhängig von dem jeweiligen Sensor, so stellt sich ebenso die Frage, bei wie vielen Sensoren eine Differenz
 als ‘zu hoch’ gelten muss, damit das gesamte Modell Alarm schlägt.

Wie in Kapitel \ref{Chapter-ML} schon angemerkt, unterscheidet sich die Güte eines Modells auch darin, welche Metrik
 benutzt wird. So folgt aus dem finanziellen und ökologischen Schaden eines Lecks, dass potentielle Lecks
 früh erkannt werden sollen; Eine niedrige Detection Time und hohe Sensitivität ist gefordert.
 Während das eigentliche Reparieren eines Rohrs in relativ geringer Zeit absolviert werden kann, ist das, was
 drum herum getan werden muss, umso zeit- und ressourcenintensiver. Vom Absperren der Straße über das Schließen
 der Ventile bis hin zum Aufgraben des Bodens bis zum Rohr können viele Stunden vergehen \cite{breakTime}. Ebenso
 kann das Zugraben des Loches viel Zeit brauchen, da sich zum Beispiel manche aufgeschüttete Bodenschichten erst
 Zeit zum sich setzen benötigen, bevor weitergemacht werden kann. Eine hohe Rate an falschen Alarmierungen, also
 eine niedrige Präzision, sollte dementsprechend auch vermieden werden. Verknüpft mit anderen Detektionsverfahren,
 wie zum Beispiel der aktiven Suche zum Überprüfen potentieller Lecks, kann der Wert auf eine optimale Präzision
 jedoch auch wieder geschmälert werden\footnote{Da diese Verfahren jedoch wie besagt auch aufwändig sind, sollte
 dennoch auf die Präzision geachtet werden.}.
